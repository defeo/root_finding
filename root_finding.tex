% !TEX encoding = UTF-8 Unicode
\documentclass{article}

\usepackage{amsmath,amssymb,amsfonts,amsmath}
\usepackage{hyperref}

\hyphenation{Berle-kamp}

%%%%%%%%%%%%%%%

\newcommand{\ff}[1]{\mathbb{F}_{#1}}
\newcommand{\fq}{\ff{q}}
\newcommand{\fqn}{\ff{q^n}}

\newcommand{\dd}{d}
\newcommand{\qq}{q}
\newcommand{\QQ}{Q}
\newcommand{\nn}{n}
\newcommand{\qn}{{\qq^\nn}}
\newcommand{\extfactfdegree}{k}
\newcommand{\extfactfsize}{\qq^{\nn \cdot \extfactfdegree}}

% if we define everything in terms of base field, extension field and
% extension field used in factorization
%
\newcommand{\basef}{\ff{\qq}}
\newcommand{\extf}{\ff{\qn}}
\newcommand{\extfactf}{\ff{\extfactfsize}}

\newcommand{\AG}{\mathrm{AG}(\qq,\nn)}

\DeclareMathOperator{\Tr}{Tr}
\DeclareMathOperator{\Ker}{Ker}
\DeclareMathOperator{\Ima}{Im} 
\DeclareMathOperator{\Decomp}{Decomp} 
\DeclareMathOperator{\Var}{Var} 
\DeclareMathOperator{\Exp}{E} 
\DeclareMathOperator{\loglog}{loglog}


% to specify the number of elements of the finite fields on which the
% trace is defined
\newcommand{\tr}[2]{\Tr_{\ff{#1}:\ff{#2}}}

% to specify the number of elements of the finite fields on which the
% trace is defined: light form
\newcommand{\trl}[2]{\Tr_{#1:#2}}

% to specify the notation of the finite fields on which the trace is
% defined
\newcommand{\trabs}[2]{\Tr_{#1:#2}}
\newcommand{\trextbase}{\trabs{\extf}{\basef}}
\newcommand{\trextfactext}{\trabs{\extfactf}{\extf}}
\newcommand{\trextfactbase}{\trabs{\extfactf}{\basef}}

\newcommand{\bigO}{O}
\newcommand{\bigOt}{\tilde{O}}
\newcommand{\smallO}{o}
\newcommand{\Mul}{\mathsf{M}}

\newcommand{\Span}{\mathbf{span}}
\newcommand{\card}[1]{\# #1}
\DeclareMathOperator{\Res}{Res}

\newcommand{\cost}[1]{\color{blue}Cost:  #1\color{black}}

%%%%%%%%%%%% Algorithms

\usepackage{float,algorithm}
\usepackage[noend]{algorithmic}
\renewcommand{\algorithmicrequire}{\textbf{Input:}}
\renewcommand{\algorithmicensure}{\textbf{Output:}}

\newcounter{algo}

\newenvironment{algorithm_noendline}[4]{\begin{center}\begin{minipage}{0.48\textwidth}
      \refstepcounter{algo}
      \label{#4}
      \sf
      \rule{\textwidth}{0.2pt}\\
      \makebox[\textwidth][c]{Algorithm~\arabic{algo}:~\textbf{#1}}\\
      \rule[0.5\baselineskip]{\textwidth}{0.2pt}\\

      \vspace{-12pt}

      \parbox{\textwidth}{\textbf{Input} #2}
      \parbox{\textwidth}{\textbf{Output} #3}

\vspace{-7pt}

      \begin{enumerate*}}{\end{enumerate*}
      \vspace{-11pt}
\end{minipage}\end{center}
}

\newenvironment{algorithm_endline}[4]{\begin{center}\begin{minipage}{0.48\textwidth}
      \refstepcounter{algo}
      \label{#4}
      \sf
      \rule{\textwidth}{0.2pt}\\
      \makebox[\textwidth][c]{Algorithm~\arabic{algo}:~\textbf{#1}}\\
      \rule[0.5\baselineskip]{\textwidth}{0.2pt}\\

      \vspace{-12pt}

      \parbox{\textwidth}{\textbf{Input} #2}
      \parbox{\textwidth}{\textbf{Output} #3}

\vspace{-7pt}

      \begin{enumerate*}}{\end{enumerate*}
      \vspace{-11pt}
      \rule{\textwidth}{0.2pt}
\end{minipage}\end{center}
%\vspace{-0.5cm}
}

\floatstyle{plain}
\newfloat{algofloat}{thp}{bla}
\floatname{algofloat}{}

%%%%%%%%%%

\newcommand{\todo}[1]{\textcolor{red}{TODO: #1}}
\newcommand{\com }{\noindent \textcolor{blue}{Commentaire Micha\"el}:}
\newcommand{\comd}{\noindent \textcolor{blue}{D\'ebut Micha\"el}:}
\newcommand{\comf}{\noindent \textcolor{blue}{:Fin Micha\"el}}




\newtheorem{Def}{Definition}
\newtheorem{Theo}{Theorem}
\newtheorem{Prop}{Proposition}
\newtheorem{Lem}{Lemma}
\newtheorem{Coro}{Corollary}

\author{Luca De Feo, Christophe Petit, Micha\"el Quisquater}

\title{Root finding}

\begin{document}

\maketitle
\begin{abstract}
  We find roots
\end{abstract}

%%%%%%%%%%%%%%%%%%%%%%%%%%%%%%%%%%%%%%%%%%%%%%%%%%%%%%%%%%%%

\section{Introduction}
\label{sec:introduction}

\begin{itemize}
\item Contexte, previous work;
\item Our contribution;
\item Technical prerequisites: basic routines, etc.
\end{itemize}

\section{Root-finding algorithms and their relationships}
\label{sec:root-find-algor}

\begin{itemize}
\item Introduction
\item High level view of main principles and links between the
  algorihtms:
  \begin{itemize}
  \item Affine geometry;
  \item Multiplicative structure;
  \item Duality: GCD vs resultant;
  \item Randomization;
  \end{itemize}
\item Comparative chart of algorithms
\end{itemize}

\subsection{Affine geometric algorithms}
\label{sec:affine-geom-algor}

\begin{itemize}
\item Affine geometry
\item BTA
\item ARM
\item SRA
\item BTRA??
\end{itemize}

\subsection{Multiplicative algorithms}
\label{sec:mult-algor}

\begin{itemize}
\item Moenk + Mignotte-Schnorr
\item LIX determinization (not real algorithm)
\end{itemize}

\subsection{Randomized variants}
\label{sec:randomized-variants}

\begin{itemize}
\item Legendre
\item BTA rand
\item BTRA rand??
\item LIX (the real ones)
\item SRA rand
\end{itemize}

\section{New variants}
\label{sec:new-variants}

\begin{itemize}
\item Hybrid algorithms
  \begin{itemize}
  \item ARM + Moenk
  \item SRA + LIX ??
  \end{itemize}
\item Other variants
\end{itemize}

\section{Implementation and experimental results}
\label{sec:impl-exper-results}

\section{Conclusion}
\label{sec:conclusion}


%%%%%%%%%%%%%%%%%%%%%%%%%%%%%%%%%%%%%%%%%%%%%%%%%%%%%%%%%%%%

\bibliography{refs}
\bibliographystyle{plain}

\end{document}



% Local Variables:
% ispell-local-dictionary:"american"
% End:


%  LocalWords:  affine subspaces linearized factorizations
%  LocalWords:  iteratively
