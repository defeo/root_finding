\documentclass{sig-alternate}

\usepackage{amsmath,amssymb,amsfonts,amsmath}
\usepackage{url}
\usepackage{color}
\usepackage{comment}

\includecomment{mycomment}

%%%%%%%%%%%%%%%

\newcommand{\ff}[1]{\mathbb{F}_{#1}}
\newcommand{\fq}{\ff{q}}
\newcommand{\fqn}{\ff{q^n}}

\newcommand{\fctdegree}{d}
\newcommand{\basefsize}{q}
\newcommand{\extfdegree}{n}
\newcommand{\extfsize}{\basefsize^\extfdegree}
\newcommand{\extfactfdegree}{k}
\newcommand{\extfactfsize}{\basefsize^{\extfdegree \cdot \extfactfdegree}}

% if we define everything in terms of base field, extension field and
% extension field used in factorization
\newcommand{\basef}{\ff{\basefsize}}
\newcommand{\extf}{\ff{\extfsize}}
\newcommand{\extfactf}{\ff{\extfactfsize}}

\DeclareMathOperator{\Tr}{Tr}
% to specify the number of elements of the finite fields on which the
% trace is defined
\newcommand{\tr}[2]{\Tr_{\ff{#1}:\ff{#2}}}

% to specify the number of elements of the finite fields on which the
% trace is defined: light form
\newcommand{\trl}[2]{\Tr_{#1:#2}}

% to specify the notation of the finite fields on which the trace is
% defined
\newcommand{\trabs}[2]{\Tr_{#1:#2}}
\newcommand{\trextbase}{\trabs{\extf}{\basef}}
\newcommand{\trextfactext}{\trabs{\extfactf}{\extf}}
\newcommand{\trextfactbase}{\trabs{\extfactf}{\basef}}

\newcommand{\bigO}{O}
\newcommand{\bigOt}{\tilde{O}}
\newcommand{\smallO}{o}
\newcommand{\Mul}{\mathsf{M}}

\newcommand{\Span}{\mathbf{span}}
\DeclareMathOperator{\Res}{Res}

\newcommand{\cost}[1]{\color{blue}Cost:  #1\color{black}}

%%%%%%%%%%%% Algorithms

\usepackage{float,algorithm,algorithmic}
\renewcommand{\algorithmicrequire}{\textbf{Input:}}
\renewcommand{\algorithmicensure}{\textbf{Output:}}

\newcounter{algo}

\newenvironment{algorithm_noendline}[4]{\small\begin{center}\begin{minipage}{0.48\textwidth}
      \refstepcounter{algo}
      \label{#4}
      \sf
      \rule{\textwidth}{0.2pt}\\
      \makebox[\textwidth][c]{Algorithm~\arabic{algo}:~\textbf{#1}}\\
      \rule[0.5\baselineskip]{\textwidth}{0.2pt}\\

      \vspace{-12pt}

      \parbox{\textwidth}{\textbf{Input} #2}
      \parbox{\textwidth}{\textbf{Output} #3}

\vspace{-7pt}

      \begin{enumerate*}}{\end{enumerate*}
      \vspace{-11pt}
\end{minipage}\end{center}
}

\newenvironment{algorithm_endline}[4]{\small\begin{center}\begin{minipage}{0.48\textwidth}
      \refstepcounter{algo}
      \label{#4}
      \sf
      \rule{\textwidth}{0.2pt}\\
      \makebox[\textwidth][c]{Algorithm~\arabic{algo}:~\textbf{#1}}\\
      \rule[0.5\baselineskip]{\textwidth}{0.2pt}\\

      \vspace{-12pt}

      \parbox{\textwidth}{\textbf{Input} #2}
      \parbox{\textwidth}{\textbf{Output} #3}

\vspace{-7pt}

      \begin{enumerate*}}{\end{enumerate*}
      \vspace{-11pt}
      \rule{\textwidth}{0.2pt}
\end{minipage}\end{center}
%\vspace{-0.5cm}
}

\floatstyle{plain}
\newfloat{algofloat}{thp}{bla}
\floatname{algofloat}{}

%%%%%%%%%%

\newcommand{\todo}[1]{\textcolor{red}{TODO: #1}}
\newcommand{\com }{\noindent \textcolor{blue}{Commentaire Micha\"el}:}

\newtheorem{Def}{Definition}
\newtheorem{Theo}{Theorem}
\newtheorem{Prop}{Proposition}
\newtheorem{Lem}{Lemma}
\newtheorem{Coro}{Corollary}

\renewcommand{\paragraph}[1]{\smallskip\noindent{{\bf \rm #1.}}}

\numberofauthors{3}
\author{
  \alignauthor Luca De Feo\\
  \affaddr{Laboratoire PRiSM}\\
  \affaddr{Universit\'e de Versailles}\\
  \email{luca.de-feo@uvsq.fr}
  \alignauthor Christophe Petit\\
  \affaddr{Crypto Group}\\
  \affaddr{University College London}\\
  \email{}
  \alignauthor Micha\"el Quisquater\\
  \affaddr{Laboratoire PRiSM}\\
  \affaddr{Universit\'e de Versailles}\\
  \email{mquis@prism.uvsq.fr}
}

\title{Deterministic root finding in small characteristic finite
  fields}

\begin{document}

\maketitle
\begin{abstract}
  The resolution of polynomial equations over finite fields has many
  applications, for example in cryptography or coding theory. Starting
  from the work of Berlekamp in the eighties, many efficient
  algorithms have been proposed to solve this problem faster, either
  asymptotically or in practice.  In this paper, we consider the case
  of polynomials over $\fqn$ where $q$ is a small number. We propose a
  new algorithm to solve this problem, merging features of Berlekamp's
  classical trace algorithm (BTA) and the recent successive resultant
  algorithm (SRA).  Similarly to BTA, our algorithm separates the
  roots of the polynomial into distinct subspaces using gcd
  computations. Similarly to SRA, it uses particular subspaces that
  are embedded in one another.

  Our algorithm has an asymptotic complexity similar to BTA and SRA on
  this type of polynomials ... It has this and this great advantages
  over these algorithms. Our Sage implementations show that it is
  great, easy to implement, very efficient, and beat previous
  algorithms on a large range of parameters...

  We also provide optimizations and a factorization version of our
  algorithm that are of further interest for SRA.
\end{abstract}

\category{F.2.1}{Theory of computation}{Analysis of algorithms and problem complexity}[Computations in finite fields]
\category{G.4}{Mathematics of computing}{Mathematical software}
\terms{Algorithms,Theory}
\keywords{Finite fields, root finding, factoring.}

%%%%%%%%%%%%%%%%%%%%%%%%%%%%%%%%%%%%%%%%%%%%%%%%%%%%%%%%%%%%
%%%%%%%%%%%%%%%%%%%%%%%%%%%%%%%%%%%%%%%%%%%%%%%%%%%%%%%%%%%%
%%%%%%%%%%%%%%%%%%%%%%%%%%%%%%%%%%%%%%%%%%%%%%%%%%%%%%%%%%%%


\section*{Notations: to remove at the end of the writing}


\noindent {\bf Finite field:}
\begin{itemize}
\item $\ff{p}$: generic form
\item $\fq$: base field 
\item  $\fqn$ : extension field  
\end{itemize}

\medskip

\noindent {\bf Abstract notation for fields}

\begin{itemize}
\item $K$: $\basef$: base field
\item $L$:  $\extf$: extension field 
\item $E$: $\extfactf$ : field extension of $L$ with degree $k$  
\item $\basefsize$
\item $\extfsize$
\item $\extfactfsize$ : useless if we don't speak about factorization
\item $\extfdegree$ : 
\item $\extfactfdegree$ : useless if we don't speak about factorization

\end{itemize}




\medskip


\noindent {\bf Trace map on finite fields:}

\begin{itemize}
\item $\tr{p^n}{p}(X)$: generic form
\item $\trl{p^n}{p}(X)$: generic light form
\item $\trabs{L}{K}(X)$: generic abstract form
\item $\trextbase(X)$ :trace from the extension field to its base field 
\item $\trextfactext(X)$ : useless if we don't speak about factorization
\item $\trextfactbase(X)$ : useless if we don't speak about factorization

\end{itemize}


\noindent {\bf Complexity of computation in finite fields:}

Ideally we should express everything in term of the cost of a multiplication in  a finite field.

\begin{itemize}
\item $\Mul(n)$: cost of polynomial multiplication
\item Distinguish conventional arithmetic and fast arithmetic. Reuse the tables of \cite{KaltofenS97} or \cite{Umans08}.
\end{itemize}




\section{Introduction}

This paper presents and compares two related algorithms for finding
roots of polynomials with coefficients in a finite field $\extf$,
where $\basefsize$ is small. The first algorithm, called the
Successive Resultants Algorithm (SRA), was already introduced by
C.~Petit in~\cite{cgUCL-P14}. We give here a fresh look at it and show
how it is related to the second one. The second algorithm, called the
Linear Decomposition Algorithm (LDA) \todo{find a better name}, is
new, and similar in spirit to Berlekamp's Trace Algorithm
(BTA)~\cite{berl70}.

Both algorithms are deterministic. They find all the roots of a
square-free, split polynomial of degree $\fctdegree$ using \todo{}
operations over $\basef$ in the worst case, and an expected \todo{} on
average. Hence, they improve over previously known deterministic
algorithms for root finding~\cite{Shoup91b} \todo{do they?}. For some
parameter ranges, they also compare favorably in practice with the
best randomized algorithms for root finding~\cite{berl70,cantor1981}.

Any algorithm for root finding can be extended to an algorithm for
factoring polynomials as mentioned in~\cite{Rabin79}. Doing so with
our algorithms does not improve over the previous literature
\todo{unless it does?}. However we argue in
Section~\ref{sec:factorization} that a partial application of this
idea yields a practical improvement to the classical Cantor-Zassenhaus
algorithm~\cite{cantor1981}.

\paragraph{Previous work} 

\paragraph{Our contribution}

\paragraph{Outline}


\section{Background and Notations}

Let $\{v_1,\ldots,v_\extfdegree\}$ be a basis of $\fqn/\fq$. To this basis, we associate the following elements for any $i\in\{0,\ldots,\extfdegree\}$ and any $r\in\fqn$: 
we define $V_i$ as the vector space generated by $\{v_1,\ldots,v_i\}$;
we define $L_i$ as the characteristic polynomial of $V_i$;
we define $\beta_i=L_i(v_{i+1})$, $\alpha_i=\beta_i^{p-1}$ and $\gamma_i=\beta_i^{-p}$;
we define $V_{i,r}$ as the afine space $V_i+r$;
we define $M_{i,r}$ as the characteristic polynomial of $V_{i,r}$;
we define $\ell_{i,r}$ as the value $L_{i,r}(r)$.

Note that the polynomials $L_i$ and $M_i$ are linearized polynomials as defined by Berlekamp~\cite[Ch. 11]{Berlekamp1984}. The values $\alpha_i$ and the polynomials $L_i$ are consistent with the SRA paper~\cite{cgUCL-P14}; they satisfy $L_i(x)=(x^p-\alpha_ix)\circ L_{i-1}(x)$.
%
Note also that $V_{i,r}=V_{i,r'}$ if and only if $r-r'\in V_i$.
%
We finally have
$\ell_{i,r}=\sum_{j=i+1}^nr_jL_i(v_{j})$
and 
$M_{i,r}(x)=L_i(x-r)=L_i(x)-\ell_{i,r}$.

\section{Reminder on SRA}

\section{A New Deterministic Root-Finding Algorithm on $\extf$}

\subsection{Preliminary definitions and notations}

\subsection{Informal description}

Our algorithm will factor $f$ into linear terms by progressively separating its roots into smaller and smaller afine subspaces. 
%
Precisely, the algorithm successively identifies all the subspaces $V_{i,r}=V_{i}+\sum_{j=i+1}^nr_jv_j$ which contain some root $r$ of $f$, from $i=n-1$ down to 0. Note that $V_{i-1,r}\subset V_{i,r}$, and that each of the subspaces $V_{0,r}$ contains a single root $r$.

Simultaneously, the algorithm will compute monic polynomials $f_{i,r}|f$ whose roots are exactly all the roots of $f$ that are contained in $V_{i,r}$. 
%
These polynomials $f_{i,r}$ are computed recursively as
%$M_{i,r}(x)=L_{i,r}(x)-\ell_{i,r}$
$$f_{i,r}(x)=\gcd(M_{i,r}(x),f(x))
=\gcd(M_{i,r}(x),f_{i+1,r}(x))$$
for each of the $q$ polynomials $M_{i,r}=L_{i,r}-\ell_{i,r}$ that divide $M_{i+1,r}$. The polynomials $f_{i,r}(x)$ that are not equal to 1 correspond to the subspaces $M_{i,r}$ which contain some roots of $f$, and are therefore kept for the next step.
%
Note that $f_{0,r}(x)=x-r$ for any root $r$.

The computation of $\gcd(M_{i,r}(x),f_{i+1,r}(x))$ is performed simultaneously for all roots using previously computed $f_{j,r}$ for $j>i$, in a way similar to multipoint evaluation algorithms.

It remains to explain how the values $\ell_{i,r}$ are computed.
The algorithm of this section stores precomputed values $L_i(v_{j})$ and maintains approximations $\sum_{j=i+1}^nr_j$ of all roots $r$, starting from the unique approximation $r=0$ when $i=n$. 
This allows to compute $\ell_{i,r}$ as the linear combinations $\sum_{j=i+1}^nr_jL_i(v_{j})$. 
An alternative approach is presented in Section~\ref{sec:ext:variant}, where the $\ell_{i,r}$ values are computed from the $\ell_{i+1,r}$ values by solving linearized degree $q$ equations similar to the ones appearing in the second step of SRA.


\subsection{Algorithm\label{sec:alg:alg}}
Like SRA, our algorithm has three main steps:
\begin{enumerate}
\setcounter{enumi}{-1}
\item In a precomputing step (dependent on the field but not on $f$), we compute and store all the $\alpha_i$ values, as well as all $L_i(v_{j})$ values for $j>i$. This is done with the formulae  
$$L_i(v_j)=L_{i-1}(v_j)^q-\alpha_{i-1}L_{i-1}(v_j), \qquad \alpha_i=(L_{i-1}(v_i))^{q-1}.$$
\item Starting from $i=1$ up to $n$, we then compute and store the polynomials
$$L_{i,0}(x):=L_i(x)\bmod f(x).$$
This can be done recursively as
$$L_{i,0}(x)=(x^q-\alpha_ix)\circ L_{i-1,0}(x)\bmod f(x)$$
\item 	Let $r=0$ and $f_{n+1,r}=f$. \\
Starting from $i=n$ down to 1, \todo{Check indices $i$ vs $i+1$ below} for each current approximation of a root $r=\sum_{i+1}^nr_jv_j$ (defining $V_{i,r}$), we recursively identify better approximations $r=\sum_{i}^nr_jv_j$ (defining $V_{i-1,r}$) as follows:
	\begin{enumerate}
	\item For all current approximations of $r$, we compute 
	$$L_{i,r}(x):=L_{i+1,r}(x)\bmod f_{i+1,r}(x).$$
	This is done by successively computing $L_{i,r}(x)\bmod f_{j,r}(x)$, from $j=n$ down to $i+1$, and re-using all intermediary computations that are common to several roots $r$.

	\item For all current approximations of $r$ and for all $r_i\in\mathbb{F}_p$, we compute 
	%$$\ell_{i,r}=L_{i,r}\left(\sum_{j=i+1}^nr_jv_j\right)$$
	%as
	$$\ell_{i,r}=\sum_{j=i+1}^nr_jL_i(v_j)$$

	\item For all current approximations of $r$ and for all $r_i\in\mathbb{F}_p$, we compute 
	$$f_{i,r}(x)=\gcd(L_{i,r}(x)-\ell_{i,r},f_{i+1,r}(x))$$
	
	\item If $f_{i,r}=1$, we deduce that $V_{i,r}$ contains no root of $f$. If $\deg f_{i,r}=1$ then $f_{i,r}=x-r$ and we deduce one root $r$. The polynomials $f_{i,r}$ with degree larger than 1 are kept for the next step. We accordingly update all current approximations of $r$ to new values $r+r_iv_i$.

	\end{enumerate}

\end{enumerate}



\subsection{Comparison with BTA}

Given a polynomial $f$, Berlekamp's trace algorithm attempts to decompose $f$ into smaller degree factors using the product equality
$$f(x)=\prod_{r_n\in\fq}\gcd\left(f(x),\Tr(\alpha x)-r_n\right)$$
for randomly chosen $\alpha\in\fqn$.
%
Let $\{v_1,\ldots,v_{\extfdegree}\}$ be a basis such that $v_i/\alpha$ has trace 0 for all $i<\extfdegree$ and $v_\extfdegree=\alpha^{-1}$. We then have $L_{\extfdegree-1}(x)=\Tr(\alpha x)$ and $M_{\extfdegree-1,r}(x)=\Tr(\alpha (x-r))=\Tr(\alpha x)-r_\extfdegree$.

From a geometric point of view, each trace computed by BTA separates the roots of $f$ into $q$ different afine sets of size $\basefsize^{\extfdegree-1}$, corresponding to the roots of $\Tr(\alpha x)-r_\extfdegree$ for all $r_\extfdegree\in\fq$. After repeating this separation process for $O(\log_\basefsize\fctdegree)$  randomly chosen $\alpha$ values, the intersections of these afine sets become singletons, each singleton containing one root of $f$. 
%
In comparison, our algorithm separates the roots of $f$ into afine spaces $V_{i,r}$ of smaller and smaller dimensions, such that $V_{i,r}\subset V_{i+1,r}$. This has computation advantages, and makes the algorithm completely deterministic.
%
\todo{We should probably elaborate more on the advantages of Michael's algorithm, but I don't see them so clearly at the moment. BTA can also be made deterministic. I think the features are similar to ours: it looses a factor $n$ in the worst case, but this factor will just be $\log d$ usually.}


\subsection{Another comparison with BTA}


Berlekamp Trace Algorithm (BTA)~\cite{berl70} is based on the observation that

$$X^{\extfsize}-X=\prod_{c \in \basef}   (\trextbase(X)-c)$$ 
or equivalently 
\begin{equation}
\label{BTA_eq0}
\alpha\cdot (X^{\extfsize}-X)=\prod_{c \in \basef}   (\trextbase(\alpha \cdot X)-c)
\end{equation}
where $\alpha \in \extf^\ast$. Therefore, if $f(X)$ is any split polynomial of $\extf[X]$, we have
\begin{equation}
\label{BTA_eq1}
f(X)=\prod_{c \in \basef} \gcd(f(X),\trextbase(\alpha \cdot X)-c)  
\end{equation}
Relation~(\ref{BTA_eq1}) leads to a non-trivial factorization of $f$ for at least one value of $\alpha$ 
belonging to the set of any basis of $\extf$ over $\basef$~\cite{berl70}  and $\bigO(\log(n))$ attempts are required on average to find one of them \cite{Menvanovans92}. The computation of each $\gcd$ in relation~(\ref{BTA_eq1}) costs $\bigO(n \cdot \log(q) \cdot M(d))$ operations in $\extf$ and we need to consider $\bigO(\log(n))$ such relations on average to find a non-trivial factorization. Therefore, 
 to split $f$ in at least two distinct factors costs on average $\bigO(n \cdot \log(n) \cdot  q \cdot \log(q) \cdot M(d))$. The process is then recursively applied on the factors. It is possible to prove that the whole decomposition of $f$ into its linear factors costs $\bigO(n  \cdot \log(n) \cdot q \cdot \log(q) \cdot M(d) \cdot \log(d))$ (see~\cite{Gathen2003}, Th 14.11). While this algorithm is very efficient on average, its time complexity is actually unbounded because of its probabilistic nature.
 

\begin{mycomment}
\com {\bf La complexit\'e 
$$\bigO(n  \cdot \log(n) \cdot q \cdot \log(q) \cdot M(d) \cdot \log(d))$$
me para\^it pessimiste. A mon avis, il y a un facteur $q$ en trop car \`a chaque \'etape on a des chances de trouver plus de deux facteurs. Je suis cependant incapable de prouver cela pour le moment. On fait on peut aussi faire de 
la multi-evaluation \`a chaque \'etage comme dans l'algorithme de cet article. Le calcul ne tient pas compte de cette observation.
}
\end{mycomment}




\subsection{Complexity analysis}

In this analysis, we measure the complexity by the number of multiplications over $\fqn$. 
%
The precomputation step of our algorithm requires $O(qn^2)$ multiplications and Step~1 requires $O(qnd\log d)$ multiplications. The iteration $n-i$ of Step~2 has a cost bounded by $O(id)$ multiplications for Step~a, $O(iqd)$ additions for Step~b and $O(qd\log d)$ multiplications for Step~c. 

The total number of multiplications for all iterations of Step~2a is however bounded by $O(d\min(d,n))\leq O(dn)$.
% worst case: lorsque l'arbre est tres desequilibre, avec 1 seule racine d'un cote et d-1 racines de l'autre, et ainsi de suite pour les sous-branches du plus gros cote. Dans ce cas, la multi-evaluation coutera O(d^2), sauf si d>n

Step~2 will involve between $\log_qd$ and $n$ steps, depending on how the roots of $f$ are distributed in the afine spaces $V_{i,r}$. There will be only s$\log_qd$ steps if all the roots are contained in distinct subspaces $V_{n-\log_qd,r}$, but there will be exactly $n$ steps if $x(x+v_1)$ divides $f$ for some $x\in\fqn$. 
%
If $f$ is uniformly randomly chosen among all polynomials with exactly $d$ roots, then the algorithm requires $O(\log_qd)$ steps on average.


\begin{Lem}
Let $f$ be uniformly randomly chosen among all polynomials with exactly $d$ roots. For any $S>\log_qd$, the probability that the algorithm of Section~\ref{sec:alg:alg} requires less than $S$ steps is $\frac{q^S!}{(q^S-d)!q^{Sd}}$. 
\end{Lem}
\begin{proof}
The algorithm needs less than $S$ steps if and only if the $d$ distinct roots belong to distinct subspaces $V_{n-S,r}=V_{n-S}+r=V_{n-S}+\sum_{j=n-S+1}^nr_jv_j$. Since there are $q^S$ distinct such subspaces and the roots are uniformly randomly distributed, this occurs with a probability $\frac{q^S!}{(q^S-d)!q^{Sd}}$.
\end{proof}
%
\noindent By the birthday paradox, there is a close to 1/2 probability to factor $f$ completely after roughly $2\log_qd$ steps. More precisely, the expected number of steps needed by the algorithm is given by the formula
$$E=\sum_{S=\lceil\log_qd\rceil}^nS\left(\frac{q^S!}{(q^S-d)!q^{Sd}}-\frac{q^{S-1}!}{(q^{S-1}-d)!q^{(S-1)d}}\right).$$

Our algorithm performs much better on average than in the worst case. On average, the second step will have a negligible time complexity $O(qd\log^2d)$ multiplications compared to the $O(qn^2)$ and $O(qnd\log d)$ multiplications required for the precomputation and first steps.
%
In the worst case, the second step will require $O(qdn^2)$ multiplications and will become the bottleneck of the algorithm unless $\log d>n$.

The algorithm requires to store $O(n^2)$ field elements as such, but this memory can be reduced to the $O(\log d^2)$ elements required to perform Step~2 on average, at the cost of recomputing additional $L_i(v_j)$ values if needed, or restarting the algorithm with a different basis. 


%%%%%%%%%%%%%%%%
%%%%%%%%%%%%%%%%
%%%%%%%%%%%%%%%%


\section{Variant and Extensions}


\subsection{Improving the worst-case complexity\label{sec:ext:variant}}

The algorithm of Section~\ref{sec:alg:alg} may perform poorly in the worst case. In this section, we present a variant of this algorithm with worst case running time comparable to the average running time.
%
The bottleneck of the algorithm of Section~\ref{sec:alg:alg} in the worst case is the computation of all $\ell_{i,r}$ from precomputed values $L_i(v_j)$.

As $\ell_{i,r}=L_i(r)$, these values can also be computed recursively by solving the equations
$$\ell_{i+1,r}=\ell_{i,r}^q-\alpha_i\ell_{i,r}.$$
%
In the average case, there will be $O(d\log d)$ such equations to solve in the course of Step~2, but in the worst case there may be up to $O(dn)$ such equations to solve.

Similar equations are solved in the second step of SRA. It is suggested in~\cite{cgUCL-P14} to solve them with a standard root-finding algorithm at a cost equivalent to $O(n)$ operations over $\fqn$.
Here, we observe that we can reduce the asymptotic cost to $O(q)$ operations. As $\alpha_i=\beta_i^{p-1}$, we can re-write the above equation as 
$$\beta_i^{-q}\ell_{i+1,r}=(\beta_i^{-1}\ell_{i,r})^q-(\beta_i^{-1}\ell_{i,r}).$$ 
Note that the values $\beta_i$ are already computed in the precomputation step of our algorithm. We can solve the first equation by first computing $\beta_i^{-q}\ell_{i+1,r}$, then solving the equation 
$$\beta_i^{-q}\ell_{i+1,r}=\ell^q-\ell$$
and finally computing $\ell_{i,r}=\beta_i\ell$ for all solutions.

If the field extension $\fqn/\fq$ is represented by a normal basis, the equation $\beta_i^{-q}\ell_{i+1,r}=\ell^q-\ell$ amounts to a bidiagonal system that can be solved in $O(n)$ operations over $\fq$, that is about $O(1)$ operations over $\fqn$.
%
Normal bases with fast arithmetic exist for all fields and can be computed efficiently~\cite{Couveignes-Lercier}.\todo{Luca: help me define efficiently}

Using this method to compute the $\ell_{i,r}$ values ensures that the second step of our algorithm requires at most $O(qnd\log d)$ multiplications in the worst case, comparable to the complexity of the first step.
%
In practice, the version presented in Section~\ref{sec:alg:alg} is simpler to implement and achieves the same asymptotic average case complexity up to constant factors. 








\subsection{Optimisation when $n$ is composite}

\subsection{Extension to several polynomials}



\section{Applications to polynomial factorisation}
\label{sec:factorization}

\section{Experimental results}

\section{Conclusion and open problems}

computing $\alpha_i$ in $O(n)$


\bibliography{refs}
\bibliographystyle{plain}

\end{document}



% Local Variables:
% ispell-local-dictionary:"american"
% End:

