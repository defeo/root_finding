% !TEX encoding = UTF-8 Unicode
\documentclass[11pt]{article}

\usepackage{sigsam, amsmath}
\usepackage{amssymb,amsfonts,amsmath,amsthm}
\usepackage{hyperref}
\usepackage{color}
\usepackage{mdwlist}

\newtheorem{Def}{Definition}
\newtheorem{Theo}{Theorem}
\newtheorem{Prop}{Proposition}
\newtheorem{Lem}{Lemma}
\newtheorem{Coro}{Corollary}

% leave as is
\issue{Vol 48, No. 3, Issue 189, September 2014}
\articlehead{ISSAC poster abstracts}
\titlehead{Deterministic root finding in finite fields}
\authorhead{Luca De Feo, Christophe Petit and Micha\"el Quisquater}
\setcounter{page}{1}

\hyphenation{Berle-kamp}

%%%%%%%%%%%%%%%

\newcommand{\ff}[1]{\mathbb{F}_{#1}}
\newcommand{\fq}{\ff{q}}
\newcommand{\fqn}{\ff{q^n}}

\newcommand{\dd}{d}
\newcommand{\qq}{q}
\newcommand{\QQ}{Q}
\newcommand{\nn}{n}
\newcommand{\qn}{{\qq^\nn}}
\newcommand{\extfactfdegree}{k}
\newcommand{\extfactfsize}{\qq^{\nn \cdot \extfactfdegree}}

% if we define everything in terms of base field, extension field and
% extension field used in factorization
%
\newcommand{\basef}{\ff{\qq}}
\newcommand{\extf}{\ff{\qn}}
\newcommand{\extfactf}{\ff{\extfactfsize}}

\newcommand{\AG}{\mathrm{AG}(\qq,\nn)}

\DeclareMathOperator{\Tr}{Tr}
\DeclareMathOperator{\Ker}{Ker}
\DeclareMathOperator{\Ima}{Im} 
\DeclareMathOperator{\Decomp}{Decomp} 
\DeclareMathOperator{\Var}{Var} 
\DeclareMathOperator{\Exp}{E} 
\DeclareMathOperator{\loglog}{loglog}


% to specify the number of elements of the finite fields on which the
% trace is defined
\newcommand{\tr}[2]{\Tr_{\ff{#1}:\ff{#2}}}

% to specify the number of elements of the finite fields on which the
% trace is defined: light form
\newcommand{\trl}[2]{\Tr_{#1:#2}}

% to specify the notation of the finite fields on which the trace is
% defined
\newcommand{\trabs}[2]{\Tr_{#1:#2}}
\newcommand{\trextbase}{\trabs{\extf}{\basef}}
\newcommand{\trextfactext}{\trabs{\extfactf}{\extf}}
\newcommand{\trextfactbase}{\trabs{\extfactf}{\basef}}

\newcommand{\bigO}{O}
\newcommand{\bigOt}{\tilde{O}}
\newcommand{\smallO}{o}
\newcommand{\Mul}{\mathsf{M}}

\newcommand{\Span}{\mathbf{span}}
\newcommand{\card}[1]{\# #1}
\DeclareMathOperator{\Res}{Res}

\newcommand{\cost}[1]{\color{blue}Cost:  #1\color{black}}

%%%%%%%%%%%% Algorithms

\usepackage{float,algorithm}
\usepackage[noend]{algorithmic}
\renewcommand{\algorithmicrequire}{\textbf{Input:}}
\renewcommand{\algorithmicensure}{\textbf{Output:}}

\newcounter{algo}

\newenvironment{algorithm_noendline}[4]{\begin{center}\begin{minipage}{0.48\textwidth}
      \refstepcounter{algo}
      \label{#4}
      \sf
      \rule{\textwidth}{0.2pt}\\
      \makebox[\textwidth][c]{Algorithm~\arabic{algo}:~\textbf{#1}}\\
      \rule[0.5\baselineskip]{\textwidth}{0.2pt}\\

      \vspace{-12pt}

      \parbox{\textwidth}{\textbf{Input} #2}
      \parbox{\textwidth}{\textbf{Output} #3}

\vspace{-7pt}

      \begin{enumerate*}}{\end{enumerate*}
      \vspace{-11pt}
\end{minipage}\end{center}
}

\newenvironment{algorithm_endline}[4]{\begin{center}\begin{minipage}{0.48\textwidth}
      \refstepcounter{algo}
      \label{#4}
      \sf
      \rule{\textwidth}{0.2pt}\\
      \makebox[\textwidth][c]{Algorithm~\arabic{algo}:~\textbf{#1}}\\
      \rule[0.5\baselineskip]{\textwidth}{0.2pt}\\

      \vspace{-12pt}

      \parbox{\textwidth}{\textbf{Input} #2}
      \parbox{\textwidth}{\textbf{Output} #3}

\vspace{-7pt}

      \begin{enumerate*}}{\end{enumerate*}
      \vspace{-11pt}
      \rule{\textwidth}{0.2pt}
\end{minipage}\end{center}
%\vspace{-0.5cm}
}

\floatstyle{plain}
\newfloat{algofloat}{thp}{bla}
\floatname{algofloat}{}


\begin{document}

\title{Deterministic root finding in finite fields}

\author{
  Luca De Feo\footnote{
    Laboratoire PRiSM,
    Universit\'e de Versailles,
    \url{luca.de-feo@uvsq.fr}},
  Christophe Petit\footnote{
    Information Security Group,
    University College London}~ and
  Micha\"el Quisquater\footnote{
    Laboratoire PRiSM,
    Universit\'e de Versailles,
    \url{mquis@prism.uvsq.fr}}
}

\maketitle
\begin{abstract}
  Finding roots of polynomials with coefficients in a finite field is
  a special instance of the polynomial factorization problem. The most
  well known algorithm for factoring and root-finding is the
  Cantor-Zassenhaus algorithm. It is a Las Vegas algorithm with
  running time polynomial in both the polynomial degree and the field
  size. Its running time is quasi-optimal for the special case of
  root-finding.

  No deterministic polynomial time algorithm for these problems is
  known in general, however several deterministic algorithms are known
  for special instances, most notably when the characteristic of the
  finite field is small.

  The goal of this poster is to review the best deterministic
  algorithms for root-finding, in a systematic way.  We present, in a
  unified way, four algorithms :
  \begin{itemize}
  \item Berlekamp's Trace Algorithm~\cite{berl70} (BTA),
  \item Moenk's Subgroup Refinement Method~\cite{Moenck77} (SRM),
  \item Menezes, van Oorschot and Vanstone's Affine Refinement
    Method~\cite{MenezesOV88,OorschotV89} (ARM), and
  \item Petit's Successive Resultants Algorithm~\cite{cgUCL-P14}
    (SRA).
  \end{itemize}
  It is the first time that these algorithms are presented together in
  a comprehensive way, and that they are rigorously analysed,
  implemented and compared to each other.

  In doing so, we obtain several new results :
  \begin{itemize}
  \item We significantly improve the complexity ARM, matching that of
    BTA and SRA.
  \item We highlight a profound duality relationship between ARM and
    SRA.
  \item We show how to combine ARM with SRM to obtain a new algorithm,
    which always performs better, and of which ARM and SRM are special
    instances. The new algorithm considerably extends the range of
    finite fields for which deterministic polynomial time root-finding
    is possible.
  \item We present several practical and asymptotic improvements to
    special instances of the algorithms.
  \end{itemize}

  Part of these results were submitted in response to the call for
  papers of ISSAC '15, but were rejected. This poster corrects some
  minor imperfections, improves the asymptotic complexities of some
  algorithms, and presents a new algorithm not previously known.
\end{abstract}

%%%%%%%%%%%%%%%%%%%%%%%%%%%%%%%%%%%%%%%%%%%%%%%%%%%%%%%%%%%%
%%%%%%%%%%%%%%%%%%%%%%%%%%%%%%%%%%%%%%%%%%%%%%%%%%%%%%%%%%%%
%%%%%%%%%%%%%%%%%%%%%%%%%%%%%%%%%%%%%%%%%%%%%%%%%%%%%%%%%%%%

\section{Introduction}

Let $q$ be a prime power, let $\mathbb{F}_{\qq^\nn}$ be the finite
field with $\qq^\nn$ elements, and let $f$ be a polynomial of degree
$\dd$ over $\mathbb{F}_{\qq^\nn}$.
%
The \emph{root-finding problem} is the problem of computing one,
several or all elements $x\in\mathbb{F}_{\qq^\nn}$ such that $f(x)=0$.
%
This problem has many applications, in particular in cryptography and
in coding theory~\cite{McEliece78}. It also has a rich history, with
many strategies proposed over the years. In this poster we review four
related algorithms for root finding exploiting the geometric structure
of $\extf$.

 
\paragraph{Previous work} The most well known algorithm for
root-finding is Cantor and Zassenhaus'~\cite{cantor1981} algorithm. It
is a Las Vegas algorithm with expected running time $\bigOt(nd\log
q)$. It is one of the most efficient and popular methods for
root-finding (and factoring) today.

The \emph{Subgroup Refinement Method} (SRM) by Moenck~\cite{Moenck77}
is a deterministic algorithm, exploiting the multiplicative structure
of $\extf$. It intersects the root set of $f$ with a partition of
$(\extf)^\ast$ into subgroups, and recursively refines the
partition. Its complexity is quasi-linear in $d$ and in the largest
factor of $q^n-1$, thus solving the root-finding problem in
deterministic polynomial time for those fields for which $q^n-1$ is
smooth.

\emph{Berlekamp Trace Algorithm} (BTA)~\cite{berl70} takes advantage
of the factorization
$$X^{\qq^n}-X=\alpha^{-1} \cdot \prod_{r \in \basef}(\tr{\qq^n}{\qq}(\alpha \cdot X)-r)\,.$$
The roots of $f$ are separated by computing GCD's with the different
factors for some random $\alpha$. This method is recursive,
probabilistic and runs in $\bigOt(qnd)$. It can be straightforwardly
de-randomized~\cite{berl70,Shoup91b}, yielding a deterministic
polynomial time algorithm, exclusively for small-characteristic finite
fields.

The \emph{Affine Refinement Method} (ARM) by Menezes, van Oorschot and
Vanstone~\cite{MenezesOV88,OorschotV89,Menvanovans92} is a
generalization of Moenck's SRM, where the geometry of $(\extf)^\ast$
is replaced by the affine geometry of $\extf$. It can also be viewed
as a deterministic variant of BTA, using the \emph{Least Affine
  Multiple method}~\cite{mBER84a}. Its complexity is polynomial (in
small characteristic), but significantly worse than BTA. For this
reason the authors only recommend using it for very narrow parameter
ranges.

Finally, Petit\cite{cgUCL-P14} recently introduced a method called the
\emph{Successive Resultants Algorithm} (SRA). The method looks
significantly different from the previous ones, using bivariate
resultants to reduce the original root-finding problem to a much
smaller one. Its complexity is $\bigOt(qnd+n^2d)$, but, as it is
stated in the original paper, it fails to be deterministic.


\paragraph{Our contribution} This poster reviews each of BTA, SRM, ARM
and SRA.  For ARM, we present a very simple modification to the
algorithm, which brings it to the same asymptotic complexity as SRA.

We modify SRA to be fully deterministic, without losing on the
asymptotic complexity. Our presentation of SRA makes it evident that
it is based on the same geometric structure as ARM, namely the affine
geometry of $\extf$. It also highlights the fact that ARM and SRA are,
in a precise sense, \emph{dual} to each other.

We also present a new algorithm, combining ARM and SRM. The new
algorithm is polynomial in $n$, in $d$, and in the largest prime
factor of $q-1$. It reduces to ARM when $q-1$ is prime, and to SRM
when $n=1$. Hence we enlarge the range of finite fields for which a
deterministic polynomial time root-finding algorithm exists.

We also present various optimizations to ARM and SRA. Namely
\begin{itemize}
\item An asymptotic improvement in $n$ when $n$ is composite,
\item A version of ARM attaining the average case complexity in the
  worse case,
\item A version of ARM reducing the cost of precomputations on
  average.
\end{itemize}

Finally, we implement each of the algorithms above, and compare them
to the state of the art, in particular with the implementations of
Cantor-Zassenaus included in NTL and Pari/GP. Our tests show that all
the algorithms are competitive, and that their performances are very
close to one another.


\bibliography{../refs}
\bibliographystyle{plain}

\end{document}



% Local Variables:
% ispell-local-dictionary:"american"
% End:


%  LocalWords:  affine subspaces linearized factorizations
%  LocalWords:  iteratively
